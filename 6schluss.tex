\chapter{Schlussfolgerung} \label{ch:conclusion}

Mit der Durchführung dieses Projekts konnten wir die Besonderheiten eines Lebenszyklus der historischen \ac{icd10gm} Fassungen von 2007 bis 2021 darstellen. Damit wurde ein \ac{db}-Schema zur Speicherung der Information des systematischen Verzeichnisses aus dem \ac{bfarm} implementiert, eine \ac{etl}-Strecke zum Import der Information in der \ac{db} wurde programmiert und eingesetzt. Am Ende der Implementierung konnten die Besonderheiten der \ac{icd10gm} in dem vorher genannten Zeitraum, nämlich Insertion, Modifikation, Löschung und Wiederverwendung mit SQL und Python analysiert werden. Damit wurden verschiedene Ursachen für die Änderungen in den Fassungen erkannt und detailliert analysiert.

\section{Weitere Nutzungen des Systems} \label{sec:future}

Das \ac{db}-Schema unserer Implementierung ist auch nützlich für Plausibilitätsabfragen, statistische Gruppierungen und Qualitätsmanagement in klinischen Systemen, so dass Übereinstimmung und Genauigkeit der Kodierungen zum Beispiel in mehrjährigen Übersichten nachvollziehbar ist. Noch dazu kann Mithilfe unserer Implementierung im Gesundheitswesen und in der Forschung die Dienstleistungserbringungen überwacht, Gesundheitstrends erkannt, Dienstleistungen entsprechend geplant und Metaanalysen durchgeführt werden.