\chapter{Schlussfolgerung} \label{ch:conclusion}

Mit der Durchführung dieses Projekts konnten wir die Besonderheiten eines Lebenszyklus der historischen \ac{icd10gm} Auffassungen von 2007 bis 2021 darstellen. Damit wurde ein \ac{db}-Schema zur Speicherung der Information des semantischen Verzeichnisses aus dem \ac{bfarm} implementiert, eine \ac{etl}-Strecke zum Import der Information in der \ac{db} wurde programmiert und durchgespielt. Am Ende der Implementierung konnten die Besonderheiten der \ac{icd10gm} in dem vorher genannten Zeitraum, nämlich Insertion, Modifikation, Löschung und Wiederverwendung in SQL und Python analysiert werden. Damit wurden verschiedene Ursachen für die Änderungen in den Auffassungen erkannt und detailliert analysiert.

\section{Weitere Nutzungen des Systems} \label{sec:future}

Das \ac{db}-Schema unserer Implementierung ist auch nützlich für Plausibilitätsabfragen, statistische Gruppierung nach Sonderverzeichnissen und Qualitätsmanagement in klinischen Systemen, sodass Übereinstimmung und Genauigkeit der Kodierung zu einer Arztabrechnung passt. Noch dazu kann das Personal des Gesundheitswesens Mithilfe unserer Implementierung Dienstleistungserbringungen überwachen, Gesundheitstrends erkennen und Dienstleistungen entsprechend planen.