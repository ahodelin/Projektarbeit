\section{\acs{icd10gm}}

Die amtliche Klassifikation zur Verschlüsselung von Diagnosen in der ambulanten (\S 295 \ac{sgb} V) und stationären (\S 301 \ac{sgb} V) Versorgung in Deutschland ist die \ac{icd10gm}. Die Versionen und Formaten davon gelten vom Anfang bis zum Ende eines Jahres und werden im Auftrag des \ac{bmg} von dem \ac{bfarm} jährlich aktualisiert und herausgegeben \cite{icd10}. 

Die \ac{icd10gm} ist eine monohierarchisch strukturierte, alphanumerische Klassifikation mit bis zu 5 Ebenen und deren Aufbau besteht aus zwei Teilen, das systematisches und alphabetisches Verzeichnis \cite{icd10}.

\subsection{Alphabetisches Verzeichnis (Alphabet)} \label{alphadir}

Die alphabetische Zuordnung des Kodes entsteht aus der gebräuchlich Diagnosentexte. Die Systematik enthält nicht alle Diagnosen des Alphabets und eine Bezeichnung dient dazu als Verzugsbezeichnung, sodass das Alphabet auch andere Diagnosenbezeichnungen enthält, die auch veralte oder ungenau sind \cite{icd10alpha}. Ein Beispiel davon ist \textit{\acl{covid}} \acs{covid}-19 (Tabelle \ref{tab:difbe}). Aus diesem Grund existiert der \ac{alphaid}. Der nutzt die Bezeichnungen als Basis für eine nicht klassifizierenden Kode \cite{icd10alpha}.

\begin{table}[ht]
	\centering
	\small
	\caption[Verschiedene Bezeichnungen von COVID-19]{Verschiedene Bezeichnungen von \ac{covid}-19 mit deren \ac{alphaid} und \ac{icd10gm} Identifikatoren}
	\label{tab:difbe}
	\begin{tabular}{|l|l|l|}
		\hline
		\rowcolor{lightgray} \ac{alphaid} & \ac{icd10gm} & Bezeichnung \\
		\hline
		I130805 & U07.1! & Coronavirus-Infektion-2019, durch Labortest nachgewiesen \\ \hline
		I130804 & U07.1! & Coronavirus-Infektion-2019, Virus nachgewiesen \\ \hline
		I130797 & U07.1! & Coronavirus-Krankheit-2019, Virus nachgewiesen \\ \hline
		I130809 & U07.1! & COVID-19-Infektion, durch Labortest nachgewiesen \\ \hline
		I130796 & U07.1! & COVID-19-Infektion, Virus nachgewiesen \\ \hline				
	\end{tabular}
\end{table}

\subsection{Systematisches Verzeichnis (Systematik)} 

Die Systematik ist eine hierarchisch geordnete Liste der vierstelligen Systematik des Kodes \cite{icd10syst, icd10systauf}. Dies Hierarchieebenen der \ac{icd10gm} sind Kapitel, Gruppe/Bereiche und Kode, nämlich Kategorie/Dreisteller, Subkategorien/Vier- und Fünfsteller \cite{icd10systauf}. Ein sehr wichtiges Aspekt bei der Nutzung der \ac{icd10gm}, die in vielen Datenquellen eines Krankenhaus nicht richtig wahrgenommen wird, ist dass, \textbf{alle Hinweise beim Kodieren immer berücksichtigt werden müssen} \cite{icd10systauf}.

Für die Durchführung und den Zweck dieses Projekts wird mit den \textbf{systematischen Verzeichnissen} der \ac{icd10gm} und deren vom \ac{bfarm} veröffentlichen Dateien und Metadaten gearbeitet. 

\subsection{Metadaten}

Das \ac{bfarm} veröffentlicht zwischen September und Dezember eines Jahres die Metadaten der neuen Fassung der \ac{icd10gm} in einer \ac{zip}-Datei mit Kodierungen und weiteren Informationen. Der Ordner Klassifikationsdateien enthält die aktuelle Version der Dateien für die Klassifikation der \ac{icd10gm} (Tabelle \ref{tab:classfiles}).

\begin{table}[ht]
	\centering
	\small
	\caption[Inhalt im Ordner Klassifikationsdateien ]{Liste der Dateien im Ordner Klassifikationsdateien. \glqq JJJJ\grqq{} stellt das Jahr der Fassung der \ac{icd10gm} dar.}
	\label{tab:classfiles}
	\begin{tabular}{|l|l|}
		\hline
		\rowcolor{lightgray} Dateiname & Information \\
		\hline 
		{\ttfamily icd10gmJJJJsyst\_gruppen.txt} &  Kapitel der \ac{icd10gm}-Systematik \\ \hline
		{\ttfamily icd10gmJJJJsyst\_kapitel.txt} & Kapitel der \ac{icd10gm}-Systematik \\ \hline
		{\ttfamily icd10gmJJJJsyst\_kodes.txt} & Kodes der \ac{icd10gm}-Systematik \\ \hline
		{\ttfamily morbl\_JJJJ.txt} & Morbiditätsliste  \\ \hline
		{\ttfamily mortl1\_JJJJ.txt} & Mortalitätsliste 1 \\ \hline
		{\ttfamily mortl1grp\_JJJJ.txt} & Gruppen der Mortalitätsliste 1 \\ \hline
		{\ttfamily mortl2\_JJJJ.txt} & Mortalitätsliste 2 \\ \hline
		{\ttfamily mortl3\_JJJJ.txt} & Mortalitätsliste 3 \\ \hline
		{\ttfamily ortl3grp\_JJJJ.txt} & Gruppen der Mortalitätsliste 3 \\ \hline
		{\ttfamily mortl4\_JJJJ.txt} & Mortalitätsliste 4 \\ \hline		
	\end{tabular}
\end{table}

Diese Dateien sind in \ac{csv}-Format mit \glqq ;\grqq{} als Trendzeichen. Die \ac{zip}-Datei enthält auch eine {\ttfamily icd10gmJAHRESVERSIONsyst\_metadaten\_ liesmich.txt} mit der Beschreibung aller Dateien und \ac{sql}-Statements für die Aufbau einer Datenbank oder eines Schemas (Tabellen mit den Kopfzeilen in rot in der Abbildung \ref{fig:reldb2}), je nach welchem \ac{rdbms} benutzt wird \cite{readmel}.

Die \ac{zip}-Datei befindet sich auf die Seite  \href{https://www.dimdi.de/dynamic/de/klassifikationen/downloads/}{Downloads} der Klassifikationen der \ac{bfarm} Webseite in der Sektion \ac{icd10gm}, aktuelle Jahresversion, \ac{icd10gm} {\ttfamily Jahr} Metadaten TXT (\acs{csv}).

%\textit{\glqq ....\grqq}´