\section{Verzeichnisse der \acs{icd10gm} Klassifizierung} \label{sec:difclass}

Die \ac{icd10gm} ist die amtliche Klassifikation zur Verschlüsselung von Diagnosen in der ambulanten (\S 295 \acsu{sgb} V) und stationären (\S 301 \ac{sgb} V) Versorgung in Deutschland, deren Versionen und Formate gelten vom Anfang bis zum Ende eines Jahres und werden im Auftrag des \ac{bmgg} von dem \ac{bfarm} jährlich aktualisiert und herausgegeben \cite{icd10}. 

Diese Klassifikation ist alphanumerisch, monohierarchisch strukturiert mit bis zu 5 Ebenen und deren Aufbau besteht aus zwei Teilen, das systematische und alphabetische Verzeichnis \cite{icd10}.

\subsection{Alphabetisches Verzeichnis (Alphabet)} \label{subsec:alphabetic}

Die alphabetische Zuordnung des Kodes entsteht aus den gewohnten Texten für die Diagnosen. Die Systematik enthält nicht alle Diagnosen des Alphabets. Deswegen dient die semantische Bezeichnung als Verzugsbezeichnung, sodass das Alphabet auch andere veraltete oder ungenaue Diagnosenbezeichnungen enthält \cite{icd10alpha}. Die Tabelle~\ref{tab:difbe} illustriert ein Beispiel dieser Ungenauigkeit anhand von der Diagnose \acl{covid} (\acs{covid}-19). Aus diesem Grund existiert der \ac{alphaid}. Der nutzt die Bezeichnungen als Basis für einen nicht klassifizierenden Kode \cite{icd10alpha}.

\begin{table}[ht]
	\centering
	\small
	\caption[Verschiedene Bezeichnungen von COVID-19]{Verschiedene Bezeichnungen von \ac{covid}-19 mit deren \ac{alphaid} und \ac{icd10gm} Identifikatoren}
	\label{tab:difbe}
	\begin{tabular}{|l|l|l|}
		\hline
		\rowcolor{lightgray} \ac{alphaid} & \ac{icd10gm} & Bezeichnung \\
		\hline
		I130805 & U07.1! & Coronavirus-Infektion-2019, durch Labortest nachgewiesen \\ \hline
		I130804 & U07.1! & Coronavirus-Infektion-2019, Virus nachgewiesen \\ \hline
		I130797 & U07.1! & Coronavirus-Krankheit-2019, Virus nachgewiesen \\ \hline
		I130809 & U07.1! & COVID-19-Infektion, durch Labortest nachgewiesen \\ \hline
		I130796 & U07.1! & COVID-19-Infektion, Virus nachgewiesen \\ \hline				
	\end{tabular}
\end{table}

\subsection{Systematisches Verzeichnis (Systematik)} \label{subsec:sistematic}

Die Systematik ist die hierarchisch geordnete Liste der \ac{icd10gm}-Kodierung \cite{icd10syst}. Diese Hierarchie enthält bis zu fünf Ebenen, diese sind in Kapitel, Gruppen, Kategorien (Kode bis zur dritten Stelle), und Subkategorien (vierstellige oder fünfstellige Kodes) gegliedert \cite{icd10systauf}.

Für die Durchführung und Zwecke dieses Projekts wird mit dem systematischen Verzeichnis gearbeitet.

\subsubsection{Kapitel} \label{subsubsec:chapters}

Der thematische Bereich aller Diagnosen ist in 22 Kapiteln, von I bis XXII, abgedeckt. Der Kern davon sind die organspezifischen Krankheiten von Kapiteln III bis XIV. Die nicht organspezifischen Krankheiten werden in den restlichen Kapiteln zusammengefasst. Jeder dargestellte Kodebereich wird durch ein bis vier Buchstaben gekennzeichnet, diese Buchstaben sind das erste Zeichen einer Kodierung \cite{icd10systauf}. Die Tabelle~\ref{tab:chapter} zeigt drei Beispiele von Kapiteln.

\begin{table}[ht]
	\centering
	\small
	\caption{Beispiele von Kapiteln.}
	\label{tab:chapter}
	\begin{tabular}{|l|l|p{8cm}|}
		\hline
		\rowcolor{lightgray} Kapitel & Kodebereich & Titel \\
		\hline 
		I &  \textsf{A09-B99} & Infektiöse Darmkrankheiten \\ \hline
		II &  \textsf{C00-D48} & Neubildungen \\ \hline
		XX &  \textsf{V01-Y98} & Äußere Ursachen von Morbidität und Mortalität \\ \hline
	\end{tabular}
\end{table}

\subsubsection{Gruppen} \label{subsubsec:groups}

Jede Gruppe besteht aus einem unteren und oberen Kodebereich, somit werden ca. 241 Gruppen aufgebaut. Die Bereiche werden durch einen dreistelligen alphanumerischen Kode repräsentiert. Der Buchstabe der unteren und oberen Kodebereiche ist immer derselbe \cite{icd10systauf}. Die Tabelle~\ref{tab:group} stellt Beispiele von drei Gruppen dar.

%\clearpage

\begin{table}[ht]
	\centering
	\small
	\caption{Beispiele von Gruppen.}
	\label{tab:group}
	\begin{tabular}{|l|l|l|p{8cm}|}
		\hline
		\rowcolor{lightgray} Gruppe von & Gruppe bis & Kapitel & Titel \\
		\hline 
		\textsf{A00} &  \textsf{A09} & I & Infektiöse Darmkrankheiten \\ \hline
		\textsf{F99} &  \textsf{F99} & V & Nicht näher bezeichnete psychische Störungen \\ \hline
		\textsf{U98} &  \textsf{U99} & XXII & Belegte und nicht belegte Schlüsselnummern \\ \hline
	\end{tabular}
\end{table}

\subsubsection{Kategorien und Subkategorien} \label{subsubsec:categorysubcat}

Kategorien oder \ac{icd10gm} Kodierungen haben in der Regel Subkategorien und sind maximal fünfstellig. Jeder Kodierung beginnt mit einem Großbuchstaben gefolgt von zwei Ziffern, die sogenannten dreistelligen Kodes. Andere haben einen Punkt direkt nach den zwei Ziffern, gefolgt von einer oder zwei weiteren Ziffern, die vier- und fünfstelligen Kodes. Manche dreistellige Kodes stehen für einzelne Diagnosen und stellen somit eine Kategorie dar. Andere besitzen Subkategorien, deren Kodes vier oder fünf Stellen haben können, und  mehrere Krankheiten mit gemeinsamen Merkmalen zusammenfassen \cite{icd10systauf, icd10kateg}. 

Einige vier- oder fünfstellige Subkategorien sind postkombinierte Kodes, das heißt, der Kode wird mit Hilfe einer Liste mit möglichen Werten für die restlichen Stellen gebildet \cite{icd10kateg}. Es gibt weitere Merkmale der Kategorie und Kodierung, aber diese werden nicht in diesem Projekt behandelt. Die Tabelle~\ref{tab:catsubcat} illustriert Beispiele von Kategorien und Subkategorien.

%\clearpage

\begin{table}[ht]
	\centering
	\small
	\caption{Beispiele von Kategorien und Subkategorien.}
	\label{tab:catsubcat}
	\begin{tabular}{|l|l|l|p{8cm}|}
		\hline
		\rowcolor{lightgray} Ebene & Stellen & Kode & Titel \\
		\hline 
		Kategorie &  drei & \textsf{B25} & Zytomegalie \\ \hline
		Subkategorie &  vier & \textsf{B25.0} & Pneumonie durch Zytomegalieviren \\ \hline
		Subkategorie &  fünf & \textsf{B25.80} & Infektion des Verdauungstraktes durch Zytomegalieviren \\ \hline
	\end{tabular}
\end{table}

\section{Metadaten} \label{sec:metadata}

Das \ac{bfarm} veröffentlicht zwischen September und Dezember eines Jahres die Metadaten der neuen Fassung der \ac{icd10gm} in einem komprimierten Ordner in einer \ac{zip}-Datei mit Unterordnern  mit Kodierungsdateien und weiteren Informationen. Der Unterordner Klassifikationsdateien enthält die Dateien der aktuellen Fassung der \ac{icd10gm} wie in der Tabelle~\ref{tab:classfiles} dargestellt wird. Diese Dateien nutzen das \ac{csv}-Format mit Semikolon als Trendzeichen. Die \ac{zip}-Datei enthält auch eine Liesmich-Datei unter dem Namen \glqq\textsf{icd10gmJJJJsyst\_metadaten\_liesmich.txt}\grqq{}, wobei \glqq\textsf{JJJJ}\grqq{} das Jahr der Fassung darstellt, mit der Beschreibung aller \ac{csv}-Dateien, Information der vorkommenden Erneuerungen in der Struktur dieser Dateien, und \ac{sql}-Statements für den Aufbau einer \ac{db}. Die \ac{zip}-Dateien der Version des laufenden Jahres und der Vorgängerversionen befinden sich auf der Seite \href{https://www.dimdi.de/dynamic/de/klassifikationen/downloads/}{Downloads} der Klassifikationen der \ac{bfarm} Webseite in der Sektion \ac{icd10gm}. Die Namen der Links zu den \ac{zip}-Dateien haben die Struktur \glqq\textsf{\ac{icd10gm} Version JJJJ}\grqq{} und die Struktur der Namen der \ac{zip}-Dateien ist \glqq\textsf{\ac{icd10gm}JJJJ.zip}\grqq{} .

%\clearpage

\begin{table}[ht]
	\centering
	\small
	\caption{Liste der Dateien im Ordner Klassifikationsdateien.}
	\label{tab:classfiles}
	\begin{tabular}{|l|l|}
		\hline
		\rowcolor{lightgray} Dateiname & Information \\
		\hline 
		\textsf{icd10gmJJJJsyst\_gruppen.txt} &  Gruppen der \ac{icd10gm}-Systematik \\ \hline
		\textsf{icd10gmJJJJsyst\_kapitel.txt} & Kapitel der \ac{icd10gm}-Systematik \\ \hline
		\textsf{icd10gmJJJJsyst\_kodes.txt} & Kodes der \ac{icd10gm}-Systematik \\ \hline
		\textsf{morbl\_JJJJ.txt} & Morbiditätsliste  \\ \hline
		\textsf{mortl1\_JJJJ.txt} & Mortalitätsliste 1 \\ \hline
		\textsf{mortl1grp\_JJJJ.txt} & Gruppen der Mortalitätsliste 1 \\ \hline
		\textsf{mortl2\_JJJJ.txt} & Mortalitätsliste 2 \\ \hline
		\textsf{mortl3\_JJJJ.txt} & Mortalitätsliste 3 \\ \hline
		\textsf{ortl3grp\_JJJJ.txt} & Gruppen der Mortalitätsliste 3 \\ \hline
		\textsf{mortl4\_JJJJ.txt} & Mortalitätsliste 4 \\ \hline
	\end{tabular}
\end{table}

%\textit{\glqq ....\grqq}´