\section{Verzeichnisse des \acs{icd10gm}}

Die amtliche Klassifikation zur Verschlüsselung von Diagnosen in der ambulanten (\S 295 \ac{sgb} V) und stationären (\S 301 \ac{sgb} V) Versorgung in Deutschland ist die \ac{icd10gm}. Die Versionen und Formaten davon gelten vom Anfang bis zum Ende eines Jahres und werden im Auftrag des \ac{bmg} von dem \ac{bfarm} jährlich aktualisiert und herausgegeben \cite{icd10}. Diese Klassifikation ist
alphanumerisch, monohierarchisch strukturiert mit bis zu 5 Ebenen und deren Aufbau besteht aus zwei Teilen, das systematische und alphabetische Verzeichnis \cite{icd10}.

\subsection{Alphabetische und systematische Verzeichnisse}

\subsection{Alphabetisches Verzeichnis (Alphabet)} \label{alphadir}

Die alphabetische Zuordnung des Kodes entsteht aus der gebräuchlich Diagnosentexte. Die Systematik enthält nicht alle Diagnosen des Alphabets und eine Bezeichnung dient dazu als Verzugsbezeichnung, sodass das Alphabet auch andere veraltete oder ungenaue Diagnosenbezeichnungen enthält \cite{icd10alpha}. Ein Beispiel dieser Ungenauigkeit ist \textsf{\acl{covid}} \acs{covid}-19 (sehen Sie die Tabelle \ref{tab:difbe}). Aus diesem Grund existiert der \ac{alphaid}. Der nutzt die Bezeichnungen als Basis für eine nicht klassifizierenden Kode \cite{icd10alpha}.

\begin{table}[ht]
	\centering
	\small
	\caption[Verschiedene Bezeichnungen von COVID-19]{Verschiedene Bezeichnungen von \ac{covid}-19 mit deren \ac{alphaid} und \ac{icd10gm} Identifikatoren}
	\label{tab:difbe}
	\begin{tabular}{|l|l|l|}
		\hline
		\rowcolor{lightgray} \ac{alphaid} & \ac{icd10gm} & Bezeichnung \\
		\hline
		I130805 & U07.1! & Coronavirus-Infektion-2019, durch Labortest nachgewiesen \\ \hline
		I130804 & U07.1! & Coronavirus-Infektion-2019, Virus nachgewiesen \\ \hline
		I130797 & U07.1! & Coronavirus-Krankheit-2019, Virus nachgewiesen \\ \hline
		I130809 & U07.1! & COVID-19-Infektion, durch Labortest nachgewiesen \\ \hline
		I130796 & U07.1! & COVID-19-Infektion, Virus nachgewiesen \\ \hline				
	\end{tabular}
\end{table}

\subsection{Systematisches Verzeichnis (Systematik)} 

Die Systematik ist eine hierarchisch geordnete Liste der vierstelligen Systematik des Kodes \cite{icd10syst}. Diese Hierarchieebenen der \ac{icd10gm} sind Kapitel, Gruppe/Bereiche und Kode, nämlich Kategorie/Dreisteller, Subkategorien/Vier- und Fünfsteller \cite{icd10systauf}. Für die Durchführung und den Zweck dieses Projekts wird mit dem systematischen Verzeichnis der \ac{icd10gm} und deren vom \ac{bfarm} veröffentlichen Dateien und Metadaten gearbeitet. 
%Ein sehr wichtiges Aspekt bei der Nutzung der \ac{icd10gm}, die in vielen Datenquellen eines Krankenhaus nicht richtig wahrgenommen wird, ist dass, \textbf{alle Hinweise beim Kodieren immer berücksichtigt werden müssen} \cite{icd10systauf}.
\section{Metadaten}

Das \ac{bfarm} veröffentlicht zwischen September und Dezember eines Jahres die Metadaten der neuen Fassung der \ac{icd10gm} in einer \ac{zip}-Datei mit Kodierungen und weiteren Informationen. Der Ordner Klassifikationsdateien enthält die aktuelle Version der Dateien für die Klassifikation der \ac{icd10gm} (sehen Sie die Tabelle \ref{tab:classfiles}). Diese Dateien sind in \ac{csv}-Format mit \glqq ;\grqq{} als Trendzeichen. 

\begin{table}[ht]
	\centering
	\small
	\caption[Inhalt im Ordner Klassifikationsdateien ]{Liste der Dateien im Ordner Klassifikationsdateien. \glqq JJJJ\grqq{} stellt das Jahr der Fassung der \ac{icd10gm} dar.}
	\label{tab:classfiles}
	\begin{tabular}{|l|l|}
		\hline
		\rowcolor{lightgray} Dateiname & Information \\
		\hline 
		\textsf{icd10gmJJJJsyst\_gruppen.txt} &  Kapitel der \ac{icd10gm}-Systematik \\ \hline
		\textsf{icd10gmJJJJsyst\_kapitel.txt} & Kapitel der \ac{icd10gm}-Systematik \\ \hline
		\textsf{icd10gmJJJJsyst\_kodes.txt} & Kodes der \ac{icd10gm}-Systematik \\ \hline
		\textsf{morbl\_JJJJ.txt} & Morbiditätsliste  \\ \hline
		\textsf{mortl1\_JJJJ.txt} & Mortalitätsliste 1 \\ \hline
		\textsf{mortl1grp\_JJJJ.txt} & Gruppen der Mortalitätsliste 1 \\ \hline
		\textsf{mortl2\_JJJJ.txt} & Mortalitätsliste 2 \\ \hline
		\textsf{mortl3\_JJJJ.txt} & Mortalitätsliste 3 \\ \hline
		\textsf{ortl3grp\_JJJJ.txt} & Gruppen der Mortalitätsliste 3 \\ \hline
		\textsf{mortl4\_JJJJ.txt} & Mortalitätsliste 4 \\ \hline		
	\end{tabular}
\end{table}

Die \ac{zip}-Datei enthält auch eine Liesmich-Datei mit der Beschreibung aller \ac{csv}-Dateien, vorkommenden Erneuerungen in der Struktur der \ac{csv}-Dateien, und \ac{sql}-Statements für den Aufbau einer Datenbank oder eines Schemas, je nach welchem \ac{rdbms} benutzt wird \cite{readmel}. Solche Tabellen sind mit Kopfzeilen in rot in der Abbildung \ref{fig:reldb2} dargestellt. Diese \ac{zip}-Dateien befinden sich auf der Seite  \href{https://www.dimdi.de/dynamic/de/klassifikationen/downloads/}{Downloads} der Klassifikationen der \ac{bfarm} Webseite in der Sektion \ac{icd10gm}. Hier befinden sich die Version des laufenden Jahres und die Vorgängerversionen. Die Namen der Links zu den \ac{zip}-Dateien haben die Struktur \textsf{\ac{icd10gm} Version JJJJ}. Die Struktur der Namen der \ac{zip}-Dateien ist \textsf{\ac{icd10gm}VersionJJJJ.zip}.

%\textit{\glqq ....\grqq}´